%%%%%%%%%%%%  Generated using docx2latex.com  %%%%%%%%%%%%%%

%%%%%%%%%%%%  v2.0.0-beta  %%%%%%%%%%%%%%

\documentclass[12pt]{article}
\usepackage{amsmath}
\usepackage{latexsym}
\usepackage{amsfonts}
\usepackage[normalem]{ulem}
\usepackage{array}
\usepackage{amssymb}
\usepackage{graphicx}
\usepackage[backend=biber,
style=numeric,
sorting=none,
isbn=false,
doi=false,
url=false,
]{biblatex}\addbibresource{bibliography.bib}

\usepackage{subfig}
\usepackage{wrapfig}
\usepackage{wasysym}
\usepackage{enumitem}
\usepackage{adjustbox}
\usepackage{ragged2e}
\usepackage[svgnames,table]{xcolor}
\usepackage{tikz}
\usepackage{longtable}
\usepackage{changepage}
\usepackage{setspace}
\usepackage{hhline}
\usepackage{multicol}
\usepackage{tabto}
\usepackage{float}
\usepackage{multirow}
\usepackage{makecell}
\usepackage{fancyhdr}
\usepackage[toc,page]{appendix}
\usepackage[hidelinks]{hyperref}
\usetikzlibrary{shapes.symbols,shapes.geometric,shadows,arrows.meta}
\tikzset{>={Latex[width=1.5mm,length=2mm]}}
\usepackage{flowchart}\usepackage[paperheight=11.0in,paperwidth=8.5in,left=1.0in,right=1.0in,top=1.0in,bottom=1.0in,headheight=1in]{geometry}
\usepackage[utf8]{inputenc}
\usepackage[T1]{fontenc}
\TabPositions{0.5in,1.0in,1.5in,2.0in,2.5in,3.0in,3.5in,4.0in,4.5in,5.0in,5.5in,6.0in,}

\urlstyle{same}


 %%%%%%%%%%%%  Set Depths for Sections  %%%%%%%%%%%%%%

% 1) Section
% 1.1) SubSection
% 1.1.1) SubSubSection
% 1.1.1.1) Paragraph
% 1.1.1.1.1) Subparagraph


\setcounter{tocdepth}{5}
\setcounter{secnumdepth}{5}


 %%%%%%%%%%%%  Set Depths for Nested Lists created by \begin{enumerate}  %%%%%%%%%%%%%%


\setlistdepth{9}
\renewlist{enumerate}{enumerate}{9}
		\setlist[enumerate,1]{label=\arabic*)}
		\setlist[enumerate,2]{label=\alph*)}
		\setlist[enumerate,3]{label=(\roman*)}
		\setlist[enumerate,4]{label=(\arabic*)}
		\setlist[enumerate,5]{label=(\Alph*)}
		\setlist[enumerate,6]{label=(\Roman*)}
		\setlist[enumerate,7]{label=\arabic*}
		\setlist[enumerate,8]{label=\alph*}
		\setlist[enumerate,9]{label=\roman*}

\renewlist{itemize}{itemize}{9}
		\setlist[itemize]{label=$\cdot$}
		\setlist[itemize,1]{label=\textbullet}
		\setlist[itemize,2]{label=$\circ$}
		\setlist[itemize,3]{label=$\ast$}
		\setlist[itemize,4]{label=$\dagger$}
		\setlist[itemize,5]{label=$\triangleright$}
		\setlist[itemize,6]{label=$\bigstar$}
		\setlist[itemize,7]{label=$\blacklozenge$}
		\setlist[itemize,8]{label=$\prime$}

\setlength{\topsep}{0pt}\setlength{\parindent}{0pt}
\renewcommand{\arraystretch}{1.3}


%%%%%%%%%%%%%%%%%%%% Document code starts here %%%%%%%%%%%%%%%%%%%%



\begin{document}
\begin{justify}
Economics of Sustainable Systems- Final Paper
\end{justify}\par

\begin{justify}
Rexon Carvalho
\end{justify}\par

\begin{justify}
December 16, 2015
\end{justify}\par


\vspace{\baselineskip}
\begin{Center}
\textbf{Economics of Photovoltaic Array}
\end{Center}\par


\vspace{\baselineskip}
\begin{justify}
\textbf{Introduction}
\end{justify}\par


\vspace{\baselineskip}
\begin{justify}
The demand for electricity from solar energy has increased to a great extend over the last couple of decades. The amount of electricity produced from solar panels, which convert the solar energy directly to electricity, has increased from 26 Megawatts in 2000 to 21,000 Megawatts in 2011. This increase in demand for solar energy is because of a decrease in technology costs (initial cost) and because of various policies favoring clean energy. In the united states the capacity weighted average installation costs of solar technology fell by about seventeen percent between 2009 and 2010 (1). Moreover, over the last two decades, the cost of manufacturing and installing a photovoltaic solar-power system has decreased by about 20 percent with every doubling of installed capacity.
\end{justify}\par


\vspace{\baselineskip}
\begin{justify}
The market for solar in the United States of America is largely influenced by federal, state and the local government incentives. These incentives include cash rebate, production based incentives, renewables portfolio standards, and federal and state tax benefits (1). These programs are enhanced by the positive characteristics of solar like little environmental impacts, immunity from fuel price fluctuation and ability to deploy solar energy at point of use. Because the cost of solar energy systems is so high, the policies must focus on encouraging cost reductions over time. Hence, as the policy incentives become more prevalent and as solar use is increasing, so has the desire to evaluate the costs of solar energy systems. 
\end{justify}\par


\vspace{\baselineskip}
\begin{justify}
On the other hand, the cost of producing electricity from conventional sources has increased as the price of conventional fuels like coal and natural gas increased. This has an immense influence on the prices of electricity in areas which large number of conventionally powered power plants. These regions include California, Italy, Japan and Spain. As a result, solar power is becoming cost competitive in these areas. For example, California combines abundant sunshine with retail electricity prices that, partly as a result of the state’s policies, are among the highest in the United States which are around 36 cents per kilowatt-hour for\ residential users. Unsubsidized solar power costs 36 cents per kilowatt-hour, but support from the California solar initiative cuts the price to 27 cents. Regulation are trying to limit the greenhouse gas emissions by rising the natural gas prices.  Also as the demand increase the need to construct more power plants will increase the cost of conventional energy. 
\end{justify}\par


\vspace{\baselineskip}
\begin{justify}
Despite the government incentives, decreasing costs and increasing demand, the solar energy still has a very tiny share in of the total energy generation. As a matter of fact, it provides less than half a percent of the total energy in the United States of America. Even though the costs of installation have decreased with respect to prices in the past, they are still more than the cost of installing a conventional power plant of a similar capacity by a fairly large margin. The relatively high cost of installation is a major obstacle for further market penetration. However, these standard measures of evaluating the cost and benefits do not take into account environmental benefits like substantially less green house gas emissions. 
\end{justify}\par


\vspace{\baselineskip}
\begin{justify}
Economic evaluation of solar must consider numerous characteristics that differentiate solar energy from the conventional methods of energy generation. First, the fuel (solar energy) is free. Hence, the operating costs associated with solar power substantially less as compared to conventional technologies. Second, the more energy produced from solar, the more energy produced from conventional fossil fuels will be displaced. This reduce the operating costs, greenhouse gas emissions and other pollutants like sulfur dioxide and various forms of nitrogen oxide. Hence, the marginal economic and environmental benefits related to solar power depend on the operating characteristics and the emission intensities of the conventional power units displaced. Third, generation of electricity from solar is not dispatchable\ generation.  The electricity cannot be dispatched at the request of power grid operators. It entirely depends on when the sun is shining. The non dispatchability of solar power generation gives rise to two issues. Solar generation is variable, with predictable changes over diurnal and seasonal cycles, and intermittent, with unpredictable changes due to cloud cover. However, variability can benefit as solar power systems generate maximum energy during the high demand hours when energy is valued greatest. Moreover, making solar energy dispatchable and reliable adds to cost as additional system reserves and backup generation is required.
\end{justify}\par


\vspace{\baselineskip}
\begin{justify}
This paper gives an overview of different technologies used to convert solar energy to electricity namely thick film crystalline silicon technologies (1), thin film technologies (2), organic technologies, and the economic and environmental cost-benefit analysis of using a thick film multi-crystalline solar energy systems over its lifetime (not including manufacturing) by analyzing parameters like Levelized Cost of Electricity Generation, Total Initial Cost, Economic benefit, Dollar Benefit from Reduction of Greenhouse Gas Emissions, Internal Rate of Return, Net Present Value and Breakeven Total Initial Cost for New York, California and Arizona. 
\end{justify}\par


\vspace{\baselineskip}
\begin{justify}
\textbf{Solar Technologies}
\end{justify}\par


\vspace{\baselineskip}
\begin{justify}
\uline{Thick Film Silicon} (6)
\end{justify}\par

\begin{justify}
This technology uses mono or multi-crystalline silicon. Crystalline silicon has a low absorption coefficient, which measures how well a material absorbs a given wavelength. This type of module therefore needs to be relatively thick (at least 100 $ \mu $ m thick) to absorb a sufficient part of the solar spectrum. This is one of the most common technology in the market as it is economical.
\end{justify}\par


\vspace{\baselineskip}
\begin{justify}
\uline{Thin Film Inorganic (6)}
\end{justify}\par

\begin{justify}
This technology uses less materials to be manufactured. Hence, it is relatively inexpensive which makes it more affordable than the thick film. However, it is not commonly used because it has lower efficiency, issues with manufacturing scale-up and yield (2).
\end{justify}\par


\vspace{\baselineskip}
\begin{justify}
\uline{Thin Film Organic}
\end{justify}\par

\begin{justify}
This technology uses organic dyes in place of silicon as semi-conductors. It has substantially lesser cost as organic dyes are inexpensive and the production process has lesser environmental impact. However, these solar cells have very less capacity factor which can be as less as 5$\%$ .
\end{justify}\par


\vspace{\baselineskip}

\vspace{\baselineskip}

\vspace{\baselineskip}
\begin{justify}
\textbf{Solar Array at Golisano Institute of Sustainability (GIS), RIT}
\end{justify}\par


\vspace{\baselineskip}
\begin{justify}
The data for New York State is collected from a 45 kW array at GIS, RIT. The data is logged with the help of WebCtrl in the building database.
\end{justify}\par


\vspace{\baselineskip}
\begin{justify}
\uline{Overview}
\end{justify}\par

\begin{justify}
The GIS building is home to three Suncase MX 60 Thick Film Silicon photovoltaic arrays: two on the upper roof, and one on the lower green roof on the fourth floor of the GIS building. Each array consists of a group of 60 multi-crystalline silicon cells, measuring 156x156mm each, and arranged in a 6x10 pattern. The upper roof arrays have the capacity to produce 20kW each and the lower roof array has a capacity of 5kW. DC to AC inverters are located in the Micro Grid Test Bed 1270 on the first floor. Each inverter connects to a 480Y/277V panel board that is separately metered and feeds back to the building power distribution system. The photovoltaic array at GIS has a capacity factor of 14$\%$ . Capacity factor is the percentage of the actual electricity generation with respect to max amount of electricity that can be generated (The percentage of actual amount of energy produced annually with respect to the amount of energy that could be produced if the array generated 45 kW i.e. name plate capacity, every hour of the day for 365 days).\  
\end{justify}\par


\vspace{\baselineskip}
\begin{justify}
The PV array displaces 47,725 kWh of grid electricity which is enough to power 6 average size houses for a year (4). By displacing the grid electricity, it contributes to greenhouse gas emission reduction of 11.4 tons of CO\textsubscript{2} equivalent per year. It would take 10 acres of of forest to remove the same quantity of CO\textsubscript{2} from the atmosphere (4). 
\end{justify}\par


\vspace{\baselineskip}
\begin{justify}
\uline{Performance}
\end{justify}\par

\begin{justify}
The WebCtrl system started logging data for the PV system electricity generation on 1st May 2013. Data from January 1, 2014 to December 31, 2014 was used to analyze the performance of the system. To calculate the monthly energy output, the hourly electricity generation in kWh/h was added over days of month which gave the electricity generation in kWh/month. Adding up the monthly generation gave electricity generated by the photovoltaic array in the year 2014. According to a report released by EIA on 8 July, 2015, in 2013 the New York state grid emitted 0.543 lb of greenhouse gases per kWh of electricity produced (5).  Using the data from EIA the greenhouse gas emissions reduction for every month in 2014 were calculated. The monthly and annual electricity generation of electricity is as stated in the following table (Table 1). 
\end{justify}\par


\vspace{\baselineskip}
\par



%%%%%%%%%%%%%%%%%%%% Table No: 1 starts here %%%%%%%%%%%%%%%%%%%%


\begin{table}[H]
 			\centering
\begin{tabular}{p{0.94in}p{1.81in}p{0.81in}p{2.19in}}
\hline
%row no:1
\multicolumn{1}{|p{0.94in}}{Month} & 
\multicolumn{1}{|p{1.81in}}{Total Generation (kWh)} & 
\multicolumn{1}{|p{0.81in}}{Peak (kW)} & 
\multicolumn{1}{|p{2.19in}|}{GhG Emission Reduction (Tons)} \\
\hhline{----}
%row no:2
\multicolumn{1}{|p{0.94in}}{January} & 
\multicolumn{1}{|p{1.81in}}{2398} & 
\multicolumn{1}{|p{0.81in}}{35.3} & 
\multicolumn{1}{|p{2.19in}|}{0.6} \\
\hhline{----}
%row no:3
\multicolumn{1}{|p{0.94in}}{February} & 
\multicolumn{1}{|p{1.81in}}{3038} & 
\multicolumn{1}{|p{0.81in}}{38.6} & 
\multicolumn{1}{|p{2.19in}|}{0.7} \\
\hhline{----}
%row no:4
\multicolumn{1}{|p{0.94in}}{March} & 
\multicolumn{1}{|p{1.81in}}{4183} & 
\multicolumn{1}{|p{0.81in}}{39.1} & 
\multicolumn{1}{|p{2.19in}|}{1.0} \\
\hhline{----}
%row no:5
\multicolumn{1}{|p{0.94in}}{April} & 
\multicolumn{1}{|p{1.81in}}{5006} & 
\multicolumn{1}{|p{0.81in}}{39} & 
\multicolumn{1}{|p{2.19in}|}{1.2} \\
\hhline{----}
%row no:6
\multicolumn{1}{|p{0.94in}}{May} & 
\multicolumn{1}{|p{1.81in}}{5740} & 
\multicolumn{1}{|p{0.81in}}{37.1} & 
\multicolumn{1}{|p{2.19in}|}{1.4} \\
\hhline{----}
%row no:7
\multicolumn{1}{|p{0.94in}}{June} & 
\multicolumn{1}{|p{1.81in}}{4805} & 
\multicolumn{1}{|p{0.81in}}{39.1} & 
\multicolumn{1}{|p{2.19in}|}{1.2} \\
\hhline{----}
%row no:8
\multicolumn{1}{|p{0.94in}}{July} & 
\multicolumn{1}{|p{1.81in}}{5361} & 
\multicolumn{1}{|p{0.81in}}{38.2} & 
\multicolumn{1}{|p{2.19in}|}{1.3} \\
\hhline{----}
%row no:9
\multicolumn{1}{|p{0.94in}}{August} & 
\multicolumn{1}{|p{1.81in}}{5752} & 
\multicolumn{1}{|p{0.81in}}{38.5} & 
\multicolumn{1}{|p{2.19in}|}{1.4} \\
\hhline{----}
%row no:10
\multicolumn{1}{|p{0.94in}}{September} & 
\multicolumn{1}{|p{1.81in}}{5334} & 
\multicolumn{1}{|p{0.81in}}{38.9} & 
\multicolumn{1}{|p{2.19in}|}{1.3} \\
\hhline{----}
%row no:11
\multicolumn{1}{|p{0.94in}}{October} & 
\multicolumn{1}{|p{1.81in}}{3192} & 
\multicolumn{1}{|p{0.81in}}{36.8} & 
\multicolumn{1}{|p{2.19in}|}{0.8} \\
\hhline{----}
%row no:12
\multicolumn{1}{|p{0.94in}}{November} & 
\multicolumn{1}{|p{1.81in}}{1469} & 
\multicolumn{1}{|p{0.81in}}{31.3} & 
\multicolumn{1}{|p{2.19in}|}{0.4} \\
\hhline{----}
%row no:13
\multicolumn{1}{|p{0.94in}}{December} & 
\multicolumn{1}{|p{1.81in}}{1448} & 
\multicolumn{1}{|p{0.81in}}{32} & 
\multicolumn{1}{|p{2.19in}|}{0.4} \\
\hhline{----}
%row no:14
\multicolumn{1}{|p{0.94in}}{Total} & 
\multicolumn{1}{|p{1.81in}}{47725.30} & 
\multicolumn{1}{|p{0.81in}}{Total} & 
\multicolumn{1}{|p{2.19in}|}{11.7} \\
\hhline{----}

\end{tabular}
 \end{table}


%%%%%%%%%%%%%%%%%%%% Table No: 1 ends here %%%%%%%%%%%%%%%%%%%%


\vspace{\baselineskip}
\begin{justify}
The performance of solar energy systems depends on various factors like the length of day, temperature and solar intensity. The performance increases with increase in length of day and solar intensity where as it decreases with increase in temperature. However, the effect of temperature is not as profound as the effect of length of day and solar intensity.
\end{justify}\par


\vspace{\baselineskip}


 %%%%%%%%%%%%  Starting New Page here %%%%%%%%%%%%%%

\newpage

\vspace{\baselineskip}\begin{justify}
\textbf{Cost of Solar Photovoltaics}
\end{justify}\par


\vspace{\baselineskip}
\begin{justify}
The major factor slowing down the market penetration of photovoltaic systems is its cost. Photovoltaic energy systems have a very high fixed cost which make the conventional energy technologies economically attractive. The cost of photovoltaic systems can be divided into two parts, (a) module cost (cost of solar panels), (b) balance of system cost. Balance of system cost includes the cost of inverters, energy storage systems, mounting hardware, labor, shipping, overhead, installer profit. 
\end{justify}\par

\begin{justify}
 \[ Total Initial Cost=Module Cost+Balance of System Cost \] 
\end{justify}\par

\begin{justify}
For a residential photovoltaic system, the balance of system cost account for almost 50$\%$  of the total initial cost. As the cost of module has decreased substantially the balance of system cost mow account to almost two-thirds of the total installed cost (6). The lifetime cost of electricity generation is the sum of total initial cost and the net present value of the operation and maintenance cost.
\end{justify}\par

\begin{justify}
 \[ Lifetime Cost of Electricity Generation=Total Intital Cost+NPV of Operation and Maintenence Cost \] 
\end{justify}\par

\begin{justify}
One of the best methods to compare different energy technologies is by calculating the levelized cost of electricity generation (LCOE) for each energy technology. Levelized cost of electricity generation is the price per unit of electricity generated in real dollars. It is the ratio of total cost of electricity generated over the lifetime of a particular technology to the total amount of electricity generated throughout the lifetime. It is given by the following formula.
\end{justify}\par

\begin{justify}
 \[ LCOE=\frac{Lifetime Cost of Electricity Generation  \left( \$ \right) }{Lifetime Electricity Generation  \left( kWh \right) } \] 
\end{justify}\par

\begin{justify}
 \[ LCOE=\frac{C_{in}+ \sum _{t=0}^{L}\frac{C_{t}}{ \left( 1+i \right) ^{t}}}{ \sum _{t=0}^{L}\frac{E_{t}}{ \left( 1+d \right) ^{t}}}~~~~~~~~~ Equation  \left( 1 \right)  \] 
\end{justify}\par

\begin{justify}
In the above equation C\textsubscript{in\textit{ }}is the total initial cost of photovoltaic system, C is the annual operation and maintenance cost, i is the discount rate, L is the lifetime of the technology, E is the amount of energy produced annually and d is the decrease in energy production per year over lifetime of the technology.
\end{justify}\par


\vspace{\baselineskip}
\begin{justify}
The photovoltaic array at GIS is a thick film multi-crystalline silicon array. The minimum and the maximum module costs of a thin film array are $\$$ 0.70/W\textsubscript{p} and $\$$ 1.78/W\textsubscript{p }respectively (6) and the minimum and maximum balance of system costs are $\$$ 2.32/W\textsubscript{p} and $\$$ 3.25/W\textsubscript{p }(6)\textsubscript{ }respectively. W\textsubscript{p} in the nameplate capacity of the photovoltaic system, which for the array at GIS is 45 kW. 
\end{justify}\par


\vspace{\baselineskip}
\begin{justify}
The following table (Table 2) shows the different costs associated with solar energy systems of 45 kW nameplate capacity, like the one at GIS over the course of its lifetime. 
\end{justify}\par


\vspace{\baselineskip}
\par



%%%%%%%%%%%%%%%%%%%% Table No: 2 starts here %%%%%%%%%%%%%%%%%%%%


\begin{table}[H]
 			\centering
\begin{tabular}{p{1.79in}p{0.86in}p{0.92in}p{0.92in}p{1.05in}}
\hline
%row no:1
\multicolumn{2}{|p{2.86in}}{Item Description} & 
\multicolumn{1}{|p{0.92in}}{Cost ($\$$ /W\textsubscript{p})} & 
\multicolumn{1}{|p{0.92in}}{Total Cost ($\$$ )} & 
\multicolumn{1}{|p{1.05in}|}{} \\
\hhline{-----}
%row no:2
\multicolumn{1}{|p{1.79in}}{\multirow{1}{*}{\begin{tabular}{p{1.79in}}Module Cost \\\end{tabular}}} & 
\multicolumn{1}{|p{0.86in}}{Minimum} & 
\multicolumn{1}{|p{0.92in}}{0.70} & 
\multicolumn{1}{|p{0.92in}}{31,500} & 
\multicolumn{1}{|p{1.05in}|}{\multirow{1}{*}{\begin{tabular}{p{1.05in}}\end{tabular}}} \\ 

\hhline{~---~}
%row no:3
\multicolumn{1}{|p{1.79in}}{} & 
\multicolumn{1}{|p{0.86in}}{Maximum} & 
\multicolumn{1}{|p{0.92in}}{1.78} & 
\multicolumn{1}{|p{0.92in}}{80,100} & 
\multicolumn{1}{|p{1.05in}|}{} \\
\hhline{-----}
%row no:4
\multicolumn{1}{|p{1.79in}}{\multirow{1}{*}{\begin{tabular}{p{1.79in}}Balance of System Cost\\\end{tabular}}} & 
\multicolumn{1}{|p{0.86in}}{Minimum} & 
\multicolumn{1}{|p{0.92in}}{2.32} & 
\multicolumn{1}{|p{0.92in}}{104,400} & 
\multicolumn{1}{|p{1.05in}|}{\multirow{1}{*}{\begin{tabular}{p{1.05in}}\end{tabular}}} \\ 

\hhline{~---~}
%row no:5
\multicolumn{1}{|p{1.79in}}{} & 
\multicolumn{1}{|p{0.86in}}{Maximum} & 
\multicolumn{1}{|p{0.92in}}{3.25} & 
\multicolumn{1}{|p{0.92in}}{146,250} & 
\multicolumn{1}{|p{1.05in}|}{} \\
\hhline{-----}
%row no:6
\multicolumn{1}{|p{1.79in}}{\multirow{1}{*}{\begin{tabular}{p{1.79in}}Total Initial Cost\\\end{tabular}}} & 
\multicolumn{1}{|p{0.86in}}{Minimum} & 
\multicolumn{1}{|p{0.92in}}{3.02} & 
\multicolumn{1}{|p{0.92in}}{135,900} & 
\multicolumn{1}{|p{1.05in}|}{\multirow{1}{*}{\begin{tabular}{p{1.05in}}\end{tabular}}} \\ 

\hhline{~---~}
%row no:7
\multicolumn{1}{|p{1.79in}}{} & 
\multicolumn{1}{|p{0.86in}}{Maximum} & 
\multicolumn{1}{|p{0.92in}}{5.03} & 
\multicolumn{1}{|p{0.92in}}{216,000} & 
\multicolumn{1}{|p{1.05in}|}{} \\
\hhline{-----}
%row no:8
\multicolumn{1}{|p{1.79in}}{\multirow{1}{*}{\begin{tabular}{p{1.79in}}Operation and Maintenance  ($\$$ )\\\end{tabular}}} & 
\multicolumn{2}{|p{1.99in}}{Inverter Replacement Cost (6)} & 
\multicolumn{1}{|p{0.92in}}{3000} & 
\multicolumn{1}{|p{1.05in}|}{Every 10 years} \\
\hhline{~----}
%row no:9
\multicolumn{1}{|p{1.79in}}{} & 
\multicolumn{2}{|p{1.99in}}{Miscellaneous} & 
\multicolumn{1}{|p{0.92in}}{1000} & 
\multicolumn{1}{|p{1.05in}|}{Annually} \\
\hhline{-----}
%row no:10
\multicolumn{1}{|p{1.79in}}{\multirow{1}{*}{\begin{tabular}{p{1.79in}}\textbf{Lifetime Cost of Electricity Generation ($\$$ )}\\\end{tabular}}} & 
\multicolumn{2}{|p{1.99in}}{Minimum} & 
\multicolumn{1}{|p{0.92in}}{\textbf{$\$$ 123,767.78 }} & 
\multicolumn{1}{|p{1.05in}|}{} \\
\hhline{~----}
%row no:11
\multicolumn{1}{|p{1.79in}}{} & 
\multicolumn{2}{|p{1.99in}}{Maximum} & 
\multicolumn{1}{|p{0.92in}}{\textbf{$\$$ 232,117.29 }} & 
\multicolumn{1}{|p{1.05in}|}{} \\
\hhline{-----}

\end{tabular}
 \end{table}


%%%%%%%%%%%%%%%%%%%% Table No: 2 ends here %%%%%%%%%%%%%%%%%%%%


\vspace{\baselineskip}
\begin{justify}
The levelized cost of electricity generation for the photovoltaic array at Golisano Institute of Sustainability (GIS) was calculated using the above initial costs costs, assuming annual maintenance cost $\$$ 1000 (the only maintenance required is cleaning the panels) considering an inflation rate of 0.2$\%$  (7), the initial cost of an inverter as $\$$ 3000 which has to be replaced every 10 years and its prices decreases by 2$\%$  every year (6), discount rate of 3$\%$  and a 0.68$\%$  decrease in production of electricity ever year (as per manufacturer specification sheet). The following table (Table 3) shows the calculated levelized cost of electricity generation for the photovoltaic array at GIS for minimum and maximum initial cost. The levelised cost of electricity generation for the photovoltaic array at GIS was calculated using real electricity generation data for year 2014 from the building database called WebCtrl and levelised cost of electricity generation for other technologies was were taken from literature (6).
\end{justify}\par


\vspace{\baselineskip}
\par



%%%%%%%%%%%%%%%%%%%% Table No: 3 starts here %%%%%%%%%%%%%%%%%%%%


\begin{table}[H]
 			\centering
\begin{tabular}{p{0.86in}p{0.58in}p{0.79in}p{0.55in}p{1.17in}p{1.4in}}
\hline
%row no:1
\multicolumn{1}{|p{0.86in}}{Technology} & 
\multicolumn{1}{|p{0.58in}}{Lifetime (Years)} & 
\multicolumn{1}{|p{0.79in}}{Total Initial Cost ($\$$ /W\textsubscript{p})} & 
\multicolumn{1}{|p{0.55in}}{Capacity Factor \par ($\%$ )} & 
\multicolumn{1}{|p{1.17in}}{Electricity Generation during first year (kWh)} & 
\multicolumn{1}{|p{1.4in}|}{Levelized Cost of Electricity Generation ($\$$ /kWh)} \\
\hhline{------}
%row no:2
\multicolumn{1}{|p{0.86in}}{\multirow{1}{*}{\begin{tabular}{p{0.86in}}Thick Film Silicon (GIS)\\\end{tabular}}} & 
\multicolumn{1}{|p{0.58in}}{30} & 
\multicolumn{1}{|p{0.79in}}{3.02} & 
\multicolumn{1}{|p{0.55in}}{14} & 
\multicolumn{1}{|p{1.17in}}{47,725} & 
\multicolumn{1}{|p{1.4in}|}{0.095} \\
\hhline{~-----}
%row no:3
\multicolumn{1}{|p{0.86in}}{} & 
\multicolumn{1}{|p{0.58in}}{30} & 
\multicolumn{1}{|p{0.79in}}{5.03} & 
\multicolumn{1}{|p{0.55in}}{14} & 
\multicolumn{1}{|p{1.17in}}{47,725} & 
\multicolumn{1}{|p{1.4in}|}{0.179} \\
\hhline{------}
%row no:4
\multicolumn{1}{|p{0.86in}}{\multirow{1}{*}{\begin{tabular}{p{0.86in}}Thin Film Inorganic (6)\\\end{tabular}}} & 
\multicolumn{1}{|p{0.58in}}{20} & 
\multicolumn{1}{|p{0.79in}}{0.76} & 
\multicolumn{1}{|p{0.55in}}{-} & 
\multicolumn{1}{|p{1.17in}}{-} & 
\multicolumn{1}{|p{1.4in}|}{0.13} \\
\hhline{~-----}
%row no:5
\multicolumn{1}{|p{0.86in}}{} & 
\multicolumn{1}{|p{0.58in}}{20} & 
\multicolumn{1}{|p{0.79in}}{1.65} & 
\multicolumn{1}{|p{0.55in}}{-} & 
\multicolumn{1}{|p{1.17in}}{-} & 
\multicolumn{1}{|p{1.4in}|}{0.40} \\
\hhline{------}
%row no:6
\multicolumn{1}{|p{0.86in}}{\multirow{1}{*}{\begin{tabular}{p{0.86in}}Thin Film Organic (6)\\\end{tabular}}} & 
\multicolumn{1}{|p{0.58in}}{5} & 
\multicolumn{1}{|p{0.79in}}{0.75} & 
\multicolumn{1}{|p{0.55in}}{-} & 
\multicolumn{1}{|p{1.17in}}{-} & 
\multicolumn{1}{|p{1.4in}|}{0.52} \\
\hhline{~-----}
%row no:7
\multicolumn{1}{|p{0.86in}}{} & 
\multicolumn{1}{|p{0.58in}}{5} & 
\multicolumn{1}{|p{0.79in}}{3.22} & 
\multicolumn{1}{|p{0.55in}}{-} & 
\multicolumn{1}{|p{1.17in}}{-} & 
\multicolumn{1}{|p{1.4in}|}{1.72} \\
\hhline{------}

\end{tabular}
 \end{table}


%%%%%%%%%%%%%%%%%%%% Table No: 3 ends here %%%%%%%%%%%%%%%%%%%%


\vspace{\baselineskip}
\begin{justify}
From the above table (Table 3) it can be concluded that the technology being used at GIS is one of the most economically viable technologies as it has the lowest levelized cost if electricity generation. As a result of being economical and having the largest market share (6), Thick Film Silicon technology is considered for this economic analysis.
\end{justify}\par



 %%%%%%%%%%%%  Starting New Page here %%%%%%%%%%%%%%

\newpage

\vspace{\baselineskip}\textbf{Economic Benefit of Solar Photovoltaics}\par


\vspace{\baselineskip}
\begin{justify}
The photovoltaic systems use no fuel to generate electricity. The only fuel required is solar radiation, which is free. Hence the photovoltaic systems have minimum operating cost. The only variable cost applicable is maintenance. The following table (Table 4) shows economic benefit of a photovoltaic system using the total initial cost, calculated above, for New York, California and Arizona.
\end{justify}\par


\vspace{\baselineskip}
\par



%%%%%%%%%%%%%%%%%%%% Table No: 4 starts here %%%%%%%%%%%%%%%%%%%%


\begin{table}[H]
 			\centering
\begin{tabular}{p{2.8in}p{0.99in}p{0.92in}p{0.99in}}
\hline
%row no:1
\multicolumn{1}{|p{2.8in}}{Location} & 
\multicolumn{1}{|p{0.99in}}{\textbf{New York}} & 
\multicolumn{1}{|p{0.92in}}{\textbf{California}} & 
\multicolumn{1}{|p{0.99in}|}{\textbf{Arizona}} \\
\hhline{----}
%row no:2
\multicolumn{1}{|p{2.8in}}{Capacity} & 
\multicolumn{1}{|p{0.99in}}{45 kW} & 
\multicolumn{1}{|p{0.92in}}{45 kW} & 
\multicolumn{1}{|p{0.99in}|}{45 kW} \\
\hhline{----}
%row no:3
\multicolumn{1}{|p{2.8in}}{Cost per W\textsubscript{p}} & 
\multicolumn{1}{|p{0.99in}}{$\$$ 5.03} & 
\multicolumn{1}{|p{0.92in}}{$\$$ 5.03} & 
\multicolumn{1}{|p{0.99in}|}{$\$$ 5.03} \\
\hhline{----}
%row no:4
\multicolumn{1}{|p{2.8in}}{Annual Electricity Generation (kWh)} & 
\multicolumn{1}{|p{0.99in}}{47,725} & 
\multicolumn{1}{|p{0.92in}}{64,980 \textsuperscript{(6)}} & 
\multicolumn{1}{|p{0.99in}|}{74,826 \textsuperscript{(6)}} \\
\hhline{----}
%row no:5
\multicolumn{1}{|p{2.8in}}{Total Electricity generation over lifetime (kWh)} & 
\multicolumn{1}{|p{0.99in}}{1,347,707.7 \par } & 
\multicolumn{1}{|p{0.92in}}{1,834,972.2 \par } & 
\multicolumn{1}{|p{0.99in}|}{2,113,013.7 \par } \\
\hhline{----}
%row no:6
\multicolumn{1}{|p{2.8in}}{\textbf{LCOE ($\$$ /kWh)}} & 
\multicolumn{1}{|p{0.99in}}{\textbf{$\$$ 0.172 }} & 
\multicolumn{1}{|p{0.92in}}{\textbf{$\$$ 0.13 }} & 
\multicolumn{1}{|p{0.99in}|}{\textbf{$\$$ 0.11 }} \\
\hhline{----}
%row no:7
\multicolumn{1}{|p{2.8in}}{Residential electricity rate (cents/kWh)} & 
\multicolumn{1}{|p{0.99in}}{18.44 \textsuperscript{(8)}} & 
\multicolumn{1}{|p{0.92in}}{18.38 \textsuperscript{(9)}} & 
\multicolumn{1}{|p{0.99in}|}{12.74 \textsuperscript{(10)}} \\
\hhline{----}
%row no:8
\multicolumn{1}{|p{2.8in}}{\textbf{Economic Benefit ($\$$ )}} & 
\multicolumn{1}{|p{0.99in}}{\textbf{$\$$ 175,381.10 }} & 
\multicolumn{1}{|p{0.92in}}{\textbf{$\$$ 238,013.26 }} & 
\multicolumn{1}{|p{0.99in}|}{\textbf{$\$$ 189,975.63 }} \\
\hhline{----}

\end{tabular}
 \end{table}


%%%%%%%%%%%%%%%%%%%% Table No: 4 ends here %%%%%%%%%%%%%%%%%%%%


\vspace{\baselineskip}
\begin{justify}
The nameplate capacity of the photovoltaic array at GIS is 45 kW. Hence the calculations for California and Arizona were also done for capacity of 45 kW. The initial cost of the photovoltaic array was assumed to be $\$$ 5.03 per W\textsubscript{p }for all three states for simplicity and to highlight the effect of the amount of energy generated and electricity prices on economic benifit from using a photovoltaic array. The annual energy generation data for New York was taken from the data of the photovoltaic array at GIS. The annual electricity generation for California and Arizona was calculated from the data mentioned in a paper by (Baker E., et al., 2013) (6). According to the paper, a 5 kW generates 7,220 kWh of electricity every year in California and 8314 kWh in Arizona. This numbers were scaled up for a 45 kW array. The total energy generation over a lifetime of 30 years was calculated considering a 0.68$\%$  decrease in electricity generation ever year as specified by the manufacturer. The levelized cost of electricity generation was calculated using Equation (1). The economic benefit is the net present value of yearly savings by displacing grid electricity. The economic benefit in different states was calculated considering a discount rate of 3$\%$ .
\end{justify}\par

\par 
 \begin{tikzpicture}


% Error occured here... ignoring it.
\begin{justify}

\end{justify}
\end{tikzpicture}

\vspace{\baselineskip}
\begin{justify}
The amount of money saved by using photovoltaic array depends on the amount of electricity generated. As different regions have different amount of solar radiation intensity the electricity generation can differ. For example, for same name plate capacity the generation in California is 36$\%$  more than that in New York because of higher solar radiation intensity. Hence the savings in California are substantially higher than those in New York. The savings also depend on the rate of displaced grid electricity. For example, the economic benefit in California is 25$\%$  more than that in Arizona as the price of electricity in California is higher.
\end{justify}\par


\vspace{\baselineskip}


 %%%%%%%%%%%%  Starting New Page here %%%%%%%%%%%%%%

\newpage

\vspace{\baselineskip}\textbf{Benefit from Reduction in GHG Emissions}\par


\vspace{\baselineskip}

\vspace{\baselineskip}
\par



%%%%%%%%%%%%%%%%%%%% Table No: 5 starts here %%%%%%%%%%%%%%%%%%%%


\begin{table}[H]
 			\centering
\begin{tabular}{p{2.98in}p{0.86in}p{0.86in}p{0.98in}}
\hline
%row no:1
\multicolumn{1}{|p{2.98in}}{Location} & 
\multicolumn{1}{|p{0.86in}}{\textbf{New York}} & 
\multicolumn{1}{|p{0.86in}}{\textbf{California}} & 
\multicolumn{1}{|p{0.98in}|}{\textbf{Arizona}} \\
\hhline{----}
%row no:2
\multicolumn{1}{|p{2.98in}}{Total Electricity generation over lifetime (kWh)} & 
\multicolumn{1}{|p{0.86in}}{1347707.7 \par } & 
\multicolumn{1}{|p{0.86in}}{1834972.2 \par } & 
\multicolumn{1}{|p{0.98in}|}{2113013.7 \par } \\
\hhline{----}
%row no:3
\multicolumn{1}{|p{2.98in}}{GHG emissions from grid electricity (kg/kWh)} & 
\multicolumn{1}{|p{0.86in}}{0.25 (11)} & 
\multicolumn{1}{|p{0.86in}}{0.29 (12)} & 
\multicolumn{1}{|p{0.98in}|}{0.49 (13)} \\
\hhline{----}
%row no:4
\multicolumn{1}{|p{2.98in}}{Reduction in GHG emission over lifetime (tons)} & 
\multicolumn{1}{|p{0.86in}}{331.6} & 
\multicolumn{1}{|p{0.86in}}{525.6} & 
\multicolumn{1}{|p{0.98in}|}{1030.2} \\
\hhline{----}
%row no:5
\multicolumn{1}{|p{2.98in}}{\textbf{Benefit from reduction in GHG emissions considering social cost of carbon $\$$ 21/ton CO\textsubscript{2 }eq ($\$$ )}} & 
\multicolumn{1}{|p{0.86in}}{\textbf{$\$$ 4,563.78} \par } & 
\multicolumn{1}{|p{0.86in}}{\textbf{$\$$ 7,234.16} \par } & 
\multicolumn{1}{|p{0.98in}|}{\textbf{$\$$ 14,178.69} \par } \\
\hhline{----}

\end{tabular}
 \end{table}


%%%%%%%%%%%%%%%%%%%% Table No: 5 ends here %%%%%%%%%%%%%%%%%%%%


\vspace{\baselineskip}
\begin{justify}
Using solar energy is not only economically beneficial as there is no operation cost involved but also environmentally beneficial as there are zero greenhouse gas emissions at point of electricity generation. The total greenhouse gas emission reduction is equal to the amount green house gas emissions if the same amount of electricity was generated by conventional generation methods connected to the grid. Mathematically it is the product of the amount of electricity generated by photovoltaic array and the carbon intensity of the grid (GHG emissions per kWh of grid electricity). The yearly value of green house gas emissions reduction was calculated considering social cost of carbon as $\$$ 21 per ton of CO\textsubscript{2 }equivalent (6). To make the calculations less complicated the social cost of carbon is assumed to be the same ($\$$ 21 per ton of CO\textsubscript{2 }equivalent) for the entire lifetime of the solar array. The total benefit of GHG emission reduction is the net present value of the yearly benefit over the lifetime for the array (30 years) with a discount rate of 3$\%$ .
\end{justify}\par


\vspace{\baselineskip}
\begin{justify}
Table 5 shows the benefits from greenhouse gas emission reduction in different states. The benefit from greenhouse gas emission reduction is strongly influenced by the amount of electricity generation by the photovoltaic array and the carbon intensity of the grid. For example, Arizona state has the highest electricity generation and the dirtiest grid of the three states. Hence the benefit from greenhouse gas emission reduction in Arizona state is the most. New York state has a relatively cleaner grid and the least electricity generation of the three states. Hence the benefit from reduction of greenhouse gas emissions is the least.
\end{justify}\par


\vspace{\baselineskip}
\begin{justify}
\textbf{Net Benefit (Net Present Value) of Using Solar Photovoltaics}
\end{justify}\par


\vspace{\baselineskip}
\begin{justify}
The total benefit from using solar energy is the sum of economic benefit and the benefit from reduction in greenhouse gas emission by displacing grid electricity. 
\end{justify}\par

\begin{justify}
 \[ Total Benefit=Economic Benefit+Benefit from GHG emission reduction  \] 
\end{justify}\par

\begin{justify}
Total cost, as mentioned above is the sum of total initial cost and net present value (NPV) of operation and maintenance cost.
\end{justify}\par

\begin{justify}
 \[ Lifetime Cost of Electricity Generation=Total Intital Cost+NPV of Operation and Maintenance Cost \] 
\end{justify}\par

\begin{justify}
The net benefit of using solar energy is simply the difference between the total benefit and total cost. It is also the NPV of the project because total initial cost, NPV of yearly economic benefit, NPV of annual benefit from reduction in greenhouse gas emission and NPV operation and maintenance is considered for calculation.
\end{justify}\par

\begin{justify}
 \[ Net Benefit=Total Benefit-Total Cost \] 
\end{justify}\par

\begin{justify}
The following table (Table 6) shows Net Benefit of using photovoltaic array in New York, California and Arizona.
\end{justify}\par


\vspace{\baselineskip}
\par



%%%%%%%%%%%%%%%%%%%% Table No: 6 starts here %%%%%%%%%%%%%%%%%%%%


\begin{table}[H]
 			\centering
\begin{tabular}{p{1.42in}p{1.42in}p{1.42in}p{1.42in}}
\hline
%row no:1
\multicolumn{1}{|p{1.42in}}{Location} & 
\multicolumn{1}{|p{1.42in}}{\textbf{New York}} & 
\multicolumn{1}{|p{1.42in}}{\textbf{California}} & 
\multicolumn{1}{|p{1.42in}|}{\textbf{Arizona}} \\
\hhline{----}
%row no:2
\multicolumn{1}{|p{1.42in}}{Economic Benefit} & 
\multicolumn{1}{|p{1.42in}}{$\$$ 175,381.10 } & 
\multicolumn{1}{|p{1.42in}}{$\$$ 238,013.26 } & 
\multicolumn{1}{|p{1.42in}|}{$\$$ 189,975.63 } \\
\hhline{----}
%row no:3
\multicolumn{1}{|p{1.42in}}{Benefit From GHG emission Reduction} & 
\multicolumn{1}{|p{1.42in}}{$\$$ 4,563.78 \par } & 
\multicolumn{1}{|p{1.42in}}{$\$$ 7,234.16 \par } & 
\multicolumn{1}{|p{1.42in}|}{$\$$ 14,178.69 \par } \\
\hhline{----}
%row no:4
\multicolumn{1}{|p{1.42in}}{Total Benefit} & 
\multicolumn{1}{|p{1.42in}}{$\$$ 179,944.88 } & 
\multicolumn{1}{|p{1.42in}}{$\$$ 245,247.42 } & 
\multicolumn{1}{|p{1.42in}|}{$\$$ 204,154.32 } \\
\hhline{----}
%row no:5
\multicolumn{1}{|p{1.42in}}{Total Cost} & 
\multicolumn{1}{|p{1.42in}}{$\$$ 232,117.29 } & 
\multicolumn{1}{|p{1.42in}}{$\$$ 232,117.29 } & 
\multicolumn{1}{|p{1.42in}|}{$\$$ 232,117.29 } \\
\hhline{----}
%row no:6
\multicolumn{1}{|p{1.42in}}{\textbf{Net Benefit}} & 
\multicolumn{1}{|p{1.42in}}{\textbf{-$\$$ 52,172.42}} & 
\multicolumn{1}{|p{1.42in}}{\textbf{$\$$ 13,130.13 }} & 
\multicolumn{1}{|p{1.42in}|}{\textbf{-$\$$ 27,962.98}} \\
\hhline{----}

\end{tabular}\caption{gure 3: Net Benefit of Using Solar Photovoltaics}
\label{tab:gure 3: Net Benefit of Using Solar Photovoltaics}

 \end{table}


%%%%%%%%%%%%%%%%%%%% Table No: 6 ends here %%%%%%%%%%%%%%%%%%%%


\vspace{\baselineskip}\par


\vspace{\baselineskip}
\begin{justify}
The above graph (Figure 3) shows the net benefit of using a solar array. The only place where using solar array is beneficial is California because, it produces large amount of energy as the solar intensity in California is high and the grid electricity is relatively expensive. It is not beneficial in Arizona because as the electricity price is low, the economic benefit decreases. It is not feasible in New York because it does not produce enough electricity to be economically or environmentally beneficial and because the current electricity grid is relatively cleaner. 
\end{justify}\par



 %%%%%%%%%%%%  Starting New Page here %%%%%%%%%%%%%%

\newpage

\vspace{\baselineskip}\begin{justify}
\textbf{Breakeven Total Initial Cost Internal and Rate of Return (IRR) }
\end{justify}\par


\vspace{\baselineskip}
\par



%%%%%%%%%%%%%%%%%%%% Table No: 7 starts here %%%%%%%%%%%%%%%%%%%%


\begin{table}[H]
 			\centering
\begin{tabular}{p{1.42in}p{1.42in}p{1.42in}p{1.42in}}
\hline
%row no:1
\multicolumn{1}{|p{1.42in}}{Location} & 
\multicolumn{1}{|p{1.42in}}{\textbf{New York}} & 
\multicolumn{1}{|p{1.42in}}{\textbf{California}} & 
\multicolumn{1}{|p{1.42in}|}{\textbf{Arizona}} \\
\hhline{----}
%row no:2
\multicolumn{1}{|p{1.42in}}{Net Benefit (NPV)} & 
\multicolumn{1}{|p{1.42in}}{-$\$$ 52,172.42} & 
\multicolumn{1}{|p{1.42in}}{$\$$ 13,130.13 } & 
\multicolumn{1}{|p{1.42in}|}{-$\$$ 27,962.98} \\
\hhline{----}
%row no:3
\multicolumn{1}{|p{1.42in}}{Internal Rate of Return ($\%$ )} & 
\multicolumn{1}{|p{1.42in}}{0.81$\%$  \par } & 
\multicolumn{1}{|p{1.42in}}{3.51$\%$  \par } & 
\multicolumn{1}{|p{1.42in}|}{1.86$\%$  \par } \\
\hhline{----}
%row no:4
\multicolumn{1}{|p{1.42in}}{\textbf{Breakeven Total Initial Cost ($\$$ /kW\textsubscript{p})}} & 
\multicolumn{1}{|p{1.42in}}{\textbf{$\$$ 4.10 } \par } & 
\multicolumn{1}{|p{1.42in}}{\textbf{$\$$ 5.73} \par } & 
\multicolumn{1}{|p{1.42in}|}{\textbf{$\$$ 4.70} \par } \\
\hhline{----}

\end{tabular}
 \end{table}


%%%%%%%%%%%%%%%%%%%% Table No: 7 ends here %%%%%%%%%%%%%%%%%%%%


\vspace{\baselineskip}
\begin{justify}
The Internal Rate of Return for different states is as shown in the above table (Table 7). It is the highest for California at 3.15$\%$ , Arizona is second at 1.86$\%$  and the least for New York at 0.18$\%$ . This is another evidence that solar energy is most feasible in California and least feasible in New York.
\end{justify}\par

\par 
 \begin{tikzpicture}

\path (3.17in,-2.81in) node [shape=rectangle,minimum height=0.21in,minimum width=4.31in,text width=3.88in,align=center]{Figure 4: Breakeven Total Initial Cost};
\begin{justify}

\end{justify}
\end{tikzpicture}

\vspace{\baselineskip}
\begin{justify}
The Breakeven Total Initial Cost is as shown in the above graph (Figure 4 and Table 7). For New York, the solar technology must be as cheap as $\$$ 4.01/kW\textsubscript{p }to breakeven, which is a 20$\%$  reduction in cost. Where as, for California it can be 13$\%$  more expensive and still breakeven.
\end{justify}\par


\vspace{\baselineskip}


 %%%%%%%%%%%%  Starting New Page here %%%%%%%%%%%%%%

\newpage

\vspace{\baselineskip}\begin{justify}
\textbf{Conclusion}
\end{justify}\par


\vspace{\baselineskip}
\begin{justify}
In conclusion, solar energy, as a result of having zero greenhouse gas emissions at point if generation, satisfies the first criteria to be a sustainable technology which is being environmentally friendly. It satisfies the second criteria of sustainability, which is being socially acceptable as people are willing to have it. However, the economic aspect of it highly dependent on the technology cost and various attributes of a region like solar radiation intensity, cost of grid electricity and carbon intensity of grid. As of today, thick film multi-crystalline silicon silicon arrays are most economical and have a levelized cost of electricity generation ranging from 0.095 to 0.179 $\$$ /kWh. 
\end{justify}\par

\par 
 \begin{tikzpicture}


% Error occured here... ignoring it.
\begin{justify}

\end{justify}
\end{tikzpicture}

\vspace{\baselineskip}
\begin{justify}
The solar technology (photovoltaic) is most feasible in California where the solar intensity is high, which result in more electricity generation; where the cost of grid electricity is high, due to which it is more economically beneficial and because it displaces grid energy which is not as clean as that of New York. It is not feasible in New York as the solar intensity is less due to which it cannot produce enough energy to be beneficial and the grid is relatively cleaner due to which it does not have as much environmental benefit as it has in Arizona. The the places where the net benefit is negative, the government should either subsidies it or make the grid electricity more expensive for deeper penetration of solar technology into the energy market. 
\end{justify}\par


\vspace{\baselineskip}
\uline{References}\par

\begin{enumerate}
	\item Barbose, G. et al., 2013. Tracking the Sun VI An Historical Summary of the Installed Price Tracking the Sun VI An Historical Summary of the Installed Price of. , (July), p.70. \par

	\item Surek, T., 2003. PROGRESS IN U . S . PHOTOVOLTAICS : In \textit{3rd World Conference on Photovoaltic Energy Conversion}. Osaka: 3rd World Conference on Photovoltaic Energy Conversion, pp. 2507–2512\par

	\item Goodrich, A., James, T. $\&$  Woodhouse, M., 2012. Utility-Scale Photovoltaic ( PV ) System Prices in the United States : Current Drivers and Cost-Reduction Opportunities Residential, Commercial , and Utility-Scale Photovoltaic ( PV ) System Prices in the United States: Current Drivers and Cost-Reduction. \textit{Technical Report NREL}, (February), p.64.\par

	\item "Greenhouse Gas Equivalencies Calculator." U.S. Environmental Protection Agency. 23 October 2015.\par

	\item New York State energy profile: \href{http://www.eia.gov/electricity/state/newyork/}{http://www.eia.gov/electricity/state/newyork/}\par

	\item Baker, E., Fowlie, M., Lemoine, D. and Reynolds, S.S., 2013. The economics of solar electricity. \textit{resource}, \textit{5}.\par

	\item Rehman, S., Bader, M.A. and Al-Moallem, S.A., 2007. Cost of solar energy generated using PV panels. \textit{Renewable and Sustainable Energy Reviews}, \textit{11}(8), pp.1843-1857.\par

	\item \href{https://www.eia.gov/state/data.cfm?sid=NY}{https://www.eia.gov/state/data.cfm?sid=NY$\#$ Prices}\par

	\item \href{https://www.eia.gov/state/data.cfm?sid=CA}{https://www.eia.gov/state/data.cfm?sid=CA}\par

	\item \href{https://www.eia.gov/state/data.cfm?sid=AZ}{https://www.eia.gov/state/data.cfm?sid=AZ}\par

	\item \href{http://www.eia.gov/electricity/state/newyork/}{http://www.eia.gov/electricity/state/newyork/}\par

	\item \href{http://www.eia.gov/electricity/state/california/index.cfm}{http://www.eia.gov/electricity/state/california/index.cfm}\par

	\item \href{http://www.eia.gov/electricity/state/arizona/index.cfm}{http://www.eia.gov/electricity/state/arizona/index.cfm}
\end{enumerate}\par


\vspace{\baselineskip}
\setlength{\parskip}{6.96pt}

\vspace{\baselineskip}

\vspace{\baselineskip}

\printbibliography
\end{document}